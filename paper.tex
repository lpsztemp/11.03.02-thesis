\documentclass{article}
\usepackage{fontspec}
\setmainfont{Times New Roman}

\title{Группа 11.03.02 --- про дипломы}

\begin{document}
\section*{Владислав Мануковский}

\section*{Сергей Ушаков}
Тема диплома: Шумовая карта ДВФУ. В ходе данной работы будет составлена шумовая карта Дальневосточного фиедерального университета. Проведены исследования...

\section*{Ярослав Лузан}
Акустические метаматериалы.В последние годы со всё более широким распространением компьютерного моделирования процессов перед исследователями открываются возможности по созданию абсолютно новых типов материалов, одним из которых являются так называемые акустические метаматериалы. Они представляют собой периодические структуры, то есть состоящие из повторяющихся элементов размером от субволнового до макроразмеров, которые дают совершенно новые, неожиданные возможности.

\section*{Владислав Кипаренко}
Анализ современных методов мониторинга объектов связи
Современные методы мониторинга объектов связи представляют собой комплексные решения, сочетающие аппаратные и программные технологии для обеспечения непрерывного контроля состояния сетей, оборудования и трафика. В основе таких систем лежат автоматизированные платформы, которые используют как традиционные методы сбора данных (например, SNMP-мониторинг), так и передовые технологии, включая анализ больших данных, машинное обучение и искусственный интеллект для прогнозирования сбоев и оптимизации работы сети. Помимо этого, активно применяются технологии удаленного мониторинга через облачные сервисы, что позволяет оперативно получать информацию о состоянии объектов связи в реальном времени из любой точки мира. Особое внимание уделяется кибербезопасности, так как современные системы мониторинга должны не только отслеживать параметры работы, но и выявлять потенциальные угрозы и аномалии, предотвращая возможные инциденты.

\section*{Ярослав Лузан}
Акустические метаматериалы.В последние годы со всё более широким распространением компьютерного моделирования процессов перед исследователями открываются возможности по созданию абсолютно новых типов материалов, одним из которых являются так называемые акустические метаматериалы. Они представляют собой периодические структуры, то есть состоящие из повторяющихся элементов размером от субволнового до макроразмеров, которые дают совершенно новые, неожиданные возможности.

\section*{Дмитрий Пакулов}
Прибор для зондирования геомагнитного поля Земли. Магни́тное по́ле Земли́ или геомагни́тное по́ле — магнитное поле, генерируемое внутриземными источниками. Предмет изучения геомагнетизма. Появилось 4,2 млрд лет назад

\section*{Матвеев Александр}
Разработка сети связи на основе АОЛС
Предполагает создание устойчивых и безопасных каналов связи на расстояниях от 50 метров до 15 километров.  

\section*{Козионов Владислав}
Беспроводная передача энергии:способы,методы,стандарты 
Диплом о передачи энергии беспроводным способом 

\end{document}
