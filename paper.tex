\documentclass{article}
\usepackage{fontspec}
\setmainfont{Times New Roman}

\title{Группа 11.03.02 --- про дипломы}

\begin{document}
\section*{Владислав Мануковский}

\section*{Ромиш Курбонов}

Темой моего ВКР является: "Разработка методов обеспечения криптографической защиты информации и принципалов на транспортном уровне стека цифровых инфокоммуникационных протоколов."
Я должен буду попытаться использовать библиотеку cryptlib в протоколе HTTP/2. Пока я на этапе перехода от HTTP/1.1 к HTTP/2.

\section*{Олеся Черняева}

\section*{Михаил Веселов}

Мой диплом о создании свёрточной нейронной сети на базе нейронной сети YOLO, которая способная распознавать и отслеживать настроенные мной объекты. Разворачиваю нейронную сеть на мини-пк Raspberry Pi 4. Также используется два сервопривода для поворота камеры по вертикали и горизонтали. На данный момент нейросеть настроена на распознавание лиц разных людей, разные модели машин, распознавание пожаров. При распознавании пожара нейросеть отправляет скриншот пожара на электронную почту с предупреждением о пожаре. Также при запуске распознавания нейросети запускается локальный стрим на сайт, который я написал под внешний вид ДВФУ.


\section*{Дмитрий Кочнев}
Локальная сеть в Спасской городской больнице
Диплом получится интересный, рассмотрим СКС и ЛВС

\section*{Артем Консевич}
Проектирование тон-зала для Творческого центра ДВФУ
В данной работе будет происходить проектирование тон-зала для Творческого центра ДВФУ. Будет проанализировано использование сэндвич-панелей для стен студии, а само  помещение будет модульный, то есть из дерева и быстрым способом постройки. Так же каждой помещение студии будет акустически проанализирован. Будет спроектирована 3d модель тон-студии.

\section*{Малютин Константин}
Исследование технологий и методов снижения удельного коэффициента поглащения электромагнитного излучения в мобильных устройствах


Излучение мобильных телефонов вызывает много обсуждений и исследований. Основное излучение, исходящее от телефона, относится к радиочастотным (РЧ) полям, которые используются для передачи данных и связи. Хотя многие исследования не обнаружили значительных рисков для здоровья при нормальном использовании, некоторые эксперты рекомендуют ограничивать время, проведенное в разговоре по телефону. Использование гарнитур и громкой связи может помочь снизить уровень воздействия. Важно оставаться информированным и следить за новыми данными о влиянии мобильных устройств на здоровье.

\end{document}
