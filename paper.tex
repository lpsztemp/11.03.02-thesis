\documentclass{article}
\usepackage{fontspec}
\setmainfont{Times New Roman}

\title{Группа 11.03.02 --- про дипломы}

\begin{document}

\section*{Владислав Мануковский}
Теоретический и экспериментальный анализ влияния операций ввода-вывода на оперативность параллельных алгоритмов свертки и дискретного преобразования Фурье

Представим, у нас есть какой-либо алгоритм. Он периодически взаимодействует с памятью, извлекая из её ячеек данные.
Когда алгоритм обращается к конкретной ячейке, он тратит на это определенное количество времени и затем адресует другую.
При этом программе может требоваться обращаться к одной ячейке памяти с набором данных, а через какое-то время обращаться к совершенно другой ячейке памяти, находящейся далеко от зоны действия алгоритма.
В противоположность этому, программе может требоваться обращение к близкорасположенным в памяти данным, адресуя ячейки на небольшом расстоянии.

В первом случае действия разница по времени может быть незначительной, но обычно в алгоритмах происходит сотни и тысячи вычислении в секунду, из-за чего незначительная задержка быстро накапливается, и скорость работы алгоритма может уменьшиться значительно.

Такая разница в подходе к реализации алгоритма и взаимодействия процессора с памятью показывает одно из важных свойств сложных программ, называющееся локальностью данных.

Кеш - это хранилище, в котором находятся многократно использованные данные, которые будут ещё использоваться в дальнейшей работе.
Например: мы в библиотеке ищем какую то необходимую информацию. Мы знаем что она находится в книге, которая находится в некоторой полке некоторого стеллажа.
Мы будем брать эту книгу с собой в руки и ставить на своё рабочее место, чтобы быстро каждый раз открывать её и обращаться к информации в ней, чем каждый раз при прочтении одного-двух предложении откладывать её на тот же стеллаж и оттуда же вытаскивать при необходимости.
Такой рабочий стол с открытой заранее книгой, где можно быстро обратиться к нужному пункту является кешем для распределенных систем.

Обычно термин кэщ применяется к вычислительным процессорам и играет в ней важную роль при обработке данных. В многоядерных процессорах кеш подразделяется на три уровня:

L1 - кеш первого уровня.
Уровень делится на кеш инструкций и кеш данных.
Располагается в ядре процессора непосредственно и используется для выполнения простых и часто повторяющихся операций.
Обычно кеш L1 имеет объем от 32 до 128 Кб на ядро.
Кеш первого уровня является самым быстрым и маленьких кешом из всех представленных уровней

L2 - кеш второго уровня.
Уровень может быть общим для нескольких ядер и выступать в качестве унифицированной памяти, но в некоторых архитектурах также может подразделяться на кеш инструкций и кеш данных.
L2 располагается рядом с ядром процессора и является промежуточным уровнем между L1 и L3. 
Обьем кеша второго уровня составляет обычно от 256 Кб до 2 Мб на ядро.

L3 - кеш третьего уровня.
В отличии от L1 и L2, привязанных к каждому ядру, L3 является общим для всех ядер процессора и хранит большие объемы данных и инструкций, использующихся различными ядрами процессора.
Объем кеша L3 может составлять в среднем от 4 до 64 Мб и более.
Это самый медленный, но самый большой по памяти уровень кеша.

Поиск необходимых данных процессор начинает с первого уровня.
Если данных в L1 нету, процессор переходит к уровню L2.
Если и там нету их, то начинается поиск в L3.
Если во всех трех уровнях данные не находятся, то поиск проходит в оперативной памяти.
Чем выше уровень кеша, тем больше задержка доступа.

Важно уметь работать с кеш-памятью, так-как это напрямую влияет на производительность процессора и скорость работы алгоритма.
Если каждое ядро процессора использует свою переменную, не пересекающуюся с переменной других ядер, и взаимодействует с одной и той же кеш линией, то это приводит к излишней перезаписи самой кеш линии на каждой итерации.
Такое явление называется ложным разделением (False sharing).
Его необходимо избегать ввиду того, что из-за ненужной инвалидации кеша скорость работы любого алгоритма снижается.

В обработке какой либо информации входного сигнала зачастую требуется раскладывать временную последовательность уровней в спектр.
С этой задачей базово справляется дискретное преобразование Фурье.
Ниже представлена формула прямого преобразования ДПФ:

\[
X_k = \sum_{n=0}^{N-1} x_n e^{-i \frac{2\pi}{N} kn}, \quad k = 0, \dots, N-1
\]

где:
- \(x_n\) — входные данные (временные отсчёты),
- \(X_k\) — выходные данные (частотные отсчёты),
- \(N\) — длина последовательности (обычно степень двойки).

ДПФ раскладывает сигнал на одиночные гармоники, у каждого из которых своя частота и амплитуда за определенный период времени.
Однако из за специфики алгоритма чем больше информации поступает на алгоритм, тем больше времени требуется на расчёт спектра.
Один из способов оптимизировать алгоритм ДПФ приводит к тому, что ДПФ становится БПФ - быстрым преобразованием Фурье.
Этот способ называется алгоритмом Кули-Тьюки.
Краткий его принцип следующий - сигнал является дискретно разделенным и имеет конечное число точек. Каждая точка разделяется на две категории - чётная и нечетная.
Для каждой точки мы проводим расчёт преобразования Фурье по синусам или по косинусам, в зависимости от того, в какой она категории.
Это позволяет уменьшить время расчёта каждой точки в два раза.
Однако можно не ограничиваться одним разделением на чётные и нечетные.
Если мы будем разделять каждую категорию ещё раз на две, а потом ещё раз и так до тех пор пока каждая точка не будет отнесена к чётной или нечётной, то таким образом сложность расчётов сократится для вычислительной машины в несколько раз.

\section*{Сергей Ушаков}

Тема диплома: Шумовая карта ДВФУ.

\section*{Ромиш Курбонов}

Темой моего ВКР является: "Разработка методов обеспечения криптографической защиты информации и принципалов на транспортном уровне стека цифровых инфокоммуникационных протоколов."
Я должен буду попытаться использовать библиотеку cryptlib в протоколе HTTP/2. Пока я на этапе перехода от HTTP/1.1 к HTTP/2.

\section*{Олеся Черняева}
В зависимости от электромагнитной обстановки выбираются различные методы для обеспечения ЭМС, которые условно делят на две группы: технические и организационные. В технические методы по обеспечению ЭМС входит ряд методов, направленных на снижение уровня создаваемых ТС помех и его восприимчивости к помехам от соседних ТС. К техническим можно отнести такие методы как экранирование, фильтрация и заземление. Ко организационным относится ряд приемов, направленных на упорядочение и ограничение работы ТС, такие как анализ работы ТС и дальнейшее выявление источников помех, пространственное распределение оборудования и распределение рабочих частот. 
Актуальность поиска новых методов по улучшению обеспечения ЭМС связана с рядом факторов: ускоренное развитие электронной техники и рост числа этой техники на производстве, в жилых домах и офисах, что приводит к ухудшению электромагнитной обстановки; активный переход радиоэлектронной техники в высокочастотный диапазон, который характеризуется меньшей длиной волны и способностью проникать в соседствующее оборудование через отверстия и швы, что непосредственно влияет на работоспособность технических средств. 

\section*{Владимир Хомяков} 
Автономная дефектоскопия трубопровода и передача информации по трубам. А про что это, я не знаю.

\section*{Денис Медведев}
Я создам антенну из метаматериалов, а потом сожгу Вовину деревню гамма изулчением. После построю кибер сарай, в котором буду обучать делать детей метаматериальные антенны, с помощю которох мы захватим галактику. !!!!!!!!!!!!!!!Make the Bussevka Greate again!!!!!!!!!!!!!!!!!!!!!
\section*{Даниил Полянцев}
Мой диплом - Создание заряда на элементе Пелетье, то есть с помощью элемента Пелетье я должен зарядить батарейку xD

\section*{Никита Шиляев}

Сравнительный обзор современных автомобильных интерфейсов передачи данных. Актуальна на сегодняшний день, до сих пор пишут статьи о появлении разных стандартов по типу CAN и CAN2

\section*{Михаил Веселов}

\section*{Яна Крапивина}

\section*{Лана Крапивина}
РАБОТКА (КУ КУ КУ КУ!!!! ): Разработка системы передачи данных между адаптивными солнечными панелями. Планируется собрать рабочую схему в лабораторных условиях ради эксперимента, так как в Росии данная тема не рассматривалась и не применялась на практике.

\section*{Ян Котенко}
- Угадай загадку: зимой и летом одним цветом?
- Знаю, знаю: Саша Белый!

\section*{Даниил Суворов}

\section*{Артем Васькив}

\section*{Дмитрий Пакулов}
Прибор для зондирования геомагнитного поля Земли. Магни́тное по́ле Земли́ или геомагни́тное по́ле — магнитное поле, генерируемое внутриземными источниками. Предмет изучения геомагнетизма. Появилось 4,2 млрд лет назад

\section*{Владислав Мануковский}
\section*{Владислав Кипаренко}
Анализ современных методов мониторинга объектов связи
Современные методы мониторинга объектов связи представляют собой комплексные решения, сочетающие аппаратные и программные технологии для обеспечения непрерывного контроля состояния сетей, оборудования и трафика. В основе таких систем лежат автоматизированные платформы, которые используют как традиционные методы сбора данных (например, SNMP-мониторинг), так и передовые технологии, включая анализ больших данных, машинное обучение и искусственный интеллект для прогнозирования сбоев и оптимизации работы сети. Помимо этого, активно применяются технологии удаленного мониторинга через облачные сервисы, что позволяет оперативно получать информацию о состоянии объектов связи в реальном времени из любой точки мира. Особое внимание уделяется кибербезопасности, так как современные системы мониторинга должны не только отслеживать параметры работы, но и выявлять потенциальные угрозы и аномалии, предотвращая возможные инциденты.

\section*{Яна Крапивина}
Тема диплома: автоматизированная система коммерческого учета электроэнергии с использованием технологии Lora.
Описание: на жилом объекте разворачивается система из антенн и базовых станций, для сбора данных с счетчиков для внесение данных с систему приложений "Умный дом".
\section*{Владислав Мануковский}
\section*{Ярослав Лузан}
Акустические метаматериалы.В последние годы со всё более широким распространением компьютерного моделирования процессов перед исследователями открываются возможности по созданию абсолютно новых типов материалов, одним из которых являются так называемые акустические метаматериалы. Они представляют собой периодические структуры, то есть состоящие из повторяющихся элементов размером от субволнового до макроразмеров, которые дают совершенно новые, неожиданные возможности.
\section*{Владислав Мануковский}
\section*{Кутырев Никита}
НУ тут что-то на кликерском
Так вот, мой деплом основан на сравнении 2-х технологий xPON и Ethernet. Нужно провести анализ технологий, выбрать нужнуюю и построить сеть интернет провайдера в DNS сити.
\section*{Малютин Константин}
Исследование технологий и методов снижения удельного коэффициента поглащения электромагнитного излучения в мобильных устройствах
\end{document}

