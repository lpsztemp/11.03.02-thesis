\documentclass{article}
\usepackage{fontspec}
\setmainfont{Times New Roman}

\title{Группа 11.03.02 --- про дипломы}

\begin{document}
\section*{Владислав Мануковский}


\section*{Дмитрий Кочнев}
Локальная сеть в Спасской городской больнице
Диплом получится интересный, рассмотрим СКС и ЛВС

\section*{Артем Консевич}
Проектирование тон-зала для Творческого центра ДВФУ
В данной работе будет происходить проектирование тон-зала для Творческого центра ДВФУ. Будет проанализировано использование сэндвич-панелей для стен студии, а само  помещение будет модульный, то есть из дерева и быстрым способом постройки. Так же каждой помещение студии будет акустически проанализирован. Будет спроектирована 3d модель тон-студии.

\section*{Малютин Константин}
Исследование технологий и методов снижения удельного коэффициента поглащения электромагнитного излучения в мобильных устройствах


Излучение мобильных телефонов вызывает много обсуждений и исследований. Основное излучение, исходящее от телефона, относится к радиочастотным (РЧ) полям, которые используются для передачи данных и связи. Хотя многие исследования не обнаружили значительных рисков для здоровья при нормальном использовании, некоторые эксперты рекомендуют ограничивать время, проведенное в разговоре по телефону. Использование гарнитур и громкой связи может помочь снизить уровень воздействия. Важно оставаться информированным и следить за новыми данными о влиянии мобильных устройств на здоровье.



\end{document}
