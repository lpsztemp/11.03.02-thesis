\documentclass{article}
\usepackage{fontspec}
\setmainfont{Times New Roman}

\title{Группа 11.03.02 --- про дипломы}

\begin{document}
\section*{Владислав Мануковский}

\section*{Ромиш Курбонов}

Темой моего ВКР является: "Разработка методов обеспечения криптографической защиты информации и принципалов на транспортном уровне стека цифровых инфокоммуникационных протоколов."
Я должен буду попытаться использовать библиотеку cryptlib в протоколе HTTP/2. Пока я на этапе перехода от HTTP/1.1 к HTTP/2.

\section*{Олеся Черняева}
В зависимости от электромагнитной обстановки выбираются различные методы для обеспечения ЭМС, которые условно делят на две группы: технические и организационные. В технические методы по обеспечению ЭМС входит ряд методов, направленных на снижение уровня создаваемых ТС помех и его восприимчивости к помехам от соседних ТС. К техническим можно отнести такие методы как экранирование, фильтрация и заземление. Ко организационным относится ряд приемов, направленных на упорядочение и ограничение работы ТС, такие как анализ работы ТС и дальнейшее выявление источников помех, пространственное распределение оборудования и распределение рабочих частот. 
Актуальность поиска новых методов по улучшению обеспечения ЭМС связана с рядом факторов: ускоренное развитие электронной техники и рост числа этой техники на производстве, в жилых домах и офисах, что приводит к ухудшению электромагнитной обстановки; активный переход радиоэлектронной техники в высокочастотный диапазон, который характеризуется меньшей длиной волны и способностью проникать в соседствующее оборудование через отверстия и швы, что непосредственно влияет на работоспособность технических средств. 

\section*{Владимир Хомяков} 
Автономная дефектоскопия трубопровода и передача информации по трубам

\section*{Денис Медведев}
Я создам антенну из метаматериалов, а потом сожгу Вовину деревню гамма изулчением. После построю кибер сарай, в котором буду обучать делать детей метаматериальные антенны, с помощю которох мы захватим галактику. !!!!!!!!!!!!!!!Make the Bussevka Greate again!!!!!!!!!!!!!!!!!!!!!
\section*{Даниил Полянцев}
Мой диплом - Создание заряда на элементе Пелетье, то есть с помощью элемента Пелетье я должен зарядить батарейку xD

\section*{Никита Шиляев}

Сравнительный обзор современных автомобильных интерфейсов передачи данных. Актуальна на сегодняшний день, до сих пор пишут статьи о появлении разных стандартов по типу CAN и CAN2

\section*{Михаил Веселов}



\end{document}
