\documentclass{article}
\usepackage{fontspec}
\setmainfont{Times New Roman}

\title{Группа 11.03.02 --- про дипломы}

\begin{document}
\section*{Владислав Мануковский}

\section*{Ромиш Курбонов}

Темой моего ВКР является: "Разработка методов обеспечения криптографической защиты информации и принципалов на транспортном уровне стека цифровых инфокоммуникационных протоколов."
Я должен буду попытаться использовать библиотеку cryptlib в протоколе HTTP/2. Пока я на этапе перехода от HTTP/1.1 к HTTP/2.

\section*{Олеся Черняева}
В зависимости от электромагнитной обстановки выбираются различные методы для обеспечения ЭМС, которые условно делят на две группы: технические и организационные. В технические методы по обеспечению ЭМС входит ряд методов, направленных на снижение уровня создаваемых ТС помех и его восприимчивости к помехам от соседних ТС. К техническим можно отнести такие методы как экранирование, фильтрация и заземление. Ко организационным относится ряд приемов, направленных на упорядочение и ограничение работы ТС, такие как анализ работы ТС и дальнейшее выявление источников помех, пространственное распределение оборудования и распределение рабочих частот. 
Актуальность поиска новых методов по улучшению обеспечения ЭМС связана с рядом факторов: ускоренное развитие электронной техники и рост числа этой техники на производстве, в жилых домах и офисах, что приводит к ухудшению электромагнитной обстановки; активный переход радиоэлектронной техники в высокочастотный диапазон, который характеризуется меньшей длиной волны и способностью проникать в соседствующее оборудование через отверстия и швы, что непосредственно влияет на работоспособность технических средств. 

\section*{Владимир Хомяков} 
Автономная дефектоскопия трубопровода и передача информации по трубам

\section*{Денис Медведев}
Я создам антенну из метаматериалов, а потом сожгу Вовину деревню гамма изулчением. После построю кибер сарай, в котором буду обучать делать детей метаматериальные антенны, с помощю которох мы захватим галактику. !!!!!!!!!!!!!!!Make the Bussevka Greate again!!!!!!!!!!!!!!!!!!!!!
\section*{Даниил Полянцев}
Мой диплом - Создание заряда на элементе Пелетье, то есть с помощью элемента Пелетье я должен зарядить батарейку xD

\section*{Никита Шиляев}

Сравнительный обзор современных автомобильных интерфейсов передачи данных. Актуальна на сегодняшний день, до сих пор пишут статьи о появлении разных стандартов по типу CAN и CAN2

\section*{Михаил Веселов}

Мой диплом о создании свёрточной нейронной сети на базе нейронной сети YOLO, которая способная распознавать и отслеживать настроенные мной объекты. Разворачиваю нейронную сеть на мини-пк Raspberry Pi 4. Также используется два сервопривода для поворота камеры по вертикали и горизонтали. На данный момент нейросеть настроена на распознавание лиц разных людей, разные модели машин, распознавание пожаров. При распознавании пожара нейросеть отправляет скриншот пожара на электронную почту с предупреждением о пожаре. Также при запуске распознавания нейросети запускается локальный стрим на сайт, который я написал под внешний вид ДВФУ.


\section*{Дмитрий Кочнев}
Локальная сеть в Спасской городской больнице
Диплом получится интересный, рассмотрим СКС и ЛВС

\section*{Артем Консевич}
Проектирование тон-зала для Творческого центра ДВФУ
В данной работе будет происходить проектирование тон-зала для Творческого центра ДВФУ. Будет проанализировано использование сэндвич-панелей для стен студии, а само  помещение будет модульный, то есть из дерева и быстрым способом постройки. Так же каждой помещение студии будет акустически проанализирован. Будет спроектирована 3d модель тон-студии.

\section*{Малютин Константин}
Исследование технологий и методов снижения удельного коэффициента поглащения электромагнитного излучения в мобильных устройствах


Излучение мобильных телефонов вызывает много обсуждений и исследований. Основное излучение, исходящее от телефона, относится к радиочастотным (РЧ) полям, которые используются для передачи данных и связи. Хотя многие исследования не обнаружили значительных рисков для здоровья при нормальном использовании, некоторые эксперты рекомендуют ограничивать время, проведенное в разговоре по телефону. Использование гарнитур и громкой связи может помочь снизить уровень воздействия. Важно оставаться информированным и следить за новыми данными о влиянии мобильных устройств на здоровье.

\end{document}


\section*{Михаил Веселов}
Тема: Нейросетевое распознавание объектов

\section*{Ян Котенко}
- Угадай загадку: зимой и летом одним цветом?
- Знаю, знаю: Саша Белый!

\section*{Данил Суворов}
Биомеханические устройства с беспроводным питанием
Целью диплома является анализ всех ныне сущесвующих методов беспроводного питания в области биомеханических устройств.


\section*{Артем Васькив}

Диплом заключается в проектирвоании и модернизации устройства диагносики мышечных реакций человека, и такде оптимизации его работы за счет улучшения кода
\section*{Васькив Артем}

Устройство диагностики мышечных реакций человека 

\section*{Владимир Рыбалка}

\end{document}

