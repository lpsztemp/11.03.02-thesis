\documentclass{article}
\usepackage{fontspec}
\setmainfont{Times New Roman}

\title{Группа 11.03.02 --- про дипломы}

\begin{document}
\section*{Владислав Мануковский}
\section*{Владислав Кипаренко}
Анализ современных методов мониторинга объектов связи
Современные методы мониторинга объектов связи представляют собой комплексные решения, сочетающие аппаратные и программные технологии для обеспечения непрерывного контроля состояния сетей, оборудования и трафика. В основе таких систем лежат автоматизированные платформы, которые используют как традиционные методы сбора данных (например, SNMP-мониторинг), так и передовые технологии, включая анализ больших данных, машинное обучение и искусственный интеллект для прогнозирования сбоев и оптимизации работы сети. Помимо этого, активно применяются технологии удаленного мониторинга через облачные сервисы, что позволяет оперативно получать информацию о состоянии объектов связи в реальном времени из любой точки мира. Особое внимание уделяется кибербезопасности, так как современные системы мониторинга должны не только отслеживать параметры работы, но и выявлять потенциальные угрозы и аномалии, предотвращая возможные инциденты.
\end{document}
