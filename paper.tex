\documentclass{article}
\usepackage{fontspec}
\setmainfont{Times New Roman}

\title{Группа 11.03.02 --- про дипломы}

\begin{document}

\section*{Владислав Мануковский}
Теоретический и экспериментальный анализ влияния операций ввода-вывода на оперативность параллельных алгоритмов свертки и дискретного преобразования Фурье

\section*{Сергей Ушаков}

Тема диплома: Шумовая карта ДВФУ.

\section*{Ромиш Курбонов}

РОМИРО БОМБАРДИРО, ИДХАН ЖЕРТВА, РОССИШ КАРТОНОВ, РАМИН КАЛГОНОВ, РОМАН КАРБОНОВ, ИЗЯ КУРБАНОВИЧ

\section*{Олеся Черняева}
В зависимости от электромагнитной обстановки выбираются различные методы для обеспечения ЭМС, которые условно делят на две группы: технические и организационные. В технические методы по обеспечению ЭМС входит ряд методов, направленных на снижение уровня создаваемых ТС помех и его восприимчивости к помехам от соседних ТС. К техническим можно отнести такие методы как экранирование, фильтрация и заземление. Ко организационным относится ряд приемов, направленных на упорядочение и ограничение работы ТС, такие как анализ работы ТС и дальнейшее выявление источников помех, пространственное распределение оборудования и распределение рабочих частот. 
Актуальность поиска новых методов по улучшению обеспечения ЭМС связана с рядом факторов: ускоренное развитие электронной техники и рост числа этой техники на производстве, в жилых домах и офисах, что приводит к ухудшению электромагнитной обстановки; активный переход радиоэлектронной техники в высокочастотный диапазон, который характеризуется меньшей длиной волны и способностью проникать в соседствующее оборудование через отверстия и швы, что непосредственно влияет на работоспособность технических средств. 

\section*{Владимир Хомяков}
РСО РСО РСО РСО РСОРСО РСО РСО РСО РСОРСО РСО РСО РСО РСОРСО РСО РСО РСО РСОРСО РСО РСО РСО РСОРСО РСО РСО РСО РСОРСО РСО РСО РСО РСОРСО РСО РСО РСО РСОРСО РСО РСО РСО РСОРСО РСО РСО РСО РСОРСО РСО РСО РСО РСОРСО РСО РСО РСО РСОРСО РСО РСО РСО РСОРСО РСО РСО РСО РСОРСО РСО РСО РСО РСОРСО РСО РСО РСО РСО

Я не хочу зачёт, не хочу диплом, и вообще хочу боли и в армию.
\section*{Денис Медведев}

\section*{Даниил Полянцев}
Мой диплом - Создание заряда на элементе Пелетье, то есть с помощью элемента Пелетье я должен зарядить батарейку xD

\section*{Никита Шиляев}

Сравнительный обзор современных автомобильных интерфейсов передачи данных. Актуальна на сегодняшний день, до сих пор пишут статьи о появлении разных стандартов по типу CAN и CAN2

\section*{Михаил Веселов}

\section*{Яна Крапивина}

\section*{Лана Крапивина}
РАБОТКА (ТУ ТУ ТУ ТУ!!!! ): Разработка системы передачи данных между адаптивными солнечными панелями. Планируется собрать рабочую схему в лабораторных условиях ради эксперимента, так как в Росии данная тема не рассматривалась и не применялась на практике.

\section*{Ян Котенко}

\section*{Даниил Суворов}

\section*{Артем Васькив}

\section*{Дмитрий Пакулов}
Прибор для зондирования геомагнитного поля Земли. Магни́тное по́ле Земли́ или геомагни́тное по́ле — магнитное поле, генерируемое внутриземными источниками. Предмет изучения геомагнетизма. Появилось 4,2 млрд лет назад
\end{document}

